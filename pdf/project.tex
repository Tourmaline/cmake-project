\documentclass[a4paper,10pt]{article}
\usepackage[utf8]{inputenc}

\usepackage{hyperref}
\usepackage{listings}
\usepackage{graphicx}

\lstset{language=C}


\title{cmake - Cross-Platform Makefile Generator}
\author{Anastasia Kruchinina}

\begin{document}

\maketitle
\tableofcontents

\begin{abstract}
CMake is a cross-platform build systems generator which allows to build software on a broad set of platforms on easier and unified way. In this report our goal is to compile the quantum chemistry code Ergo using CMake. In addition, we discover abilities of testing tools CTest and CDash.

\vskip 10pt

The project github repository: 

\url{https://github.com/Tourmaline/cmake-project.git}.

\end{abstract}

\section{Introduction to CMake}

CMake is the cross-platform, open-source build system. CMake generates native makefiles and workspaces that can be used in many different compiler environments. On Linux systems the default build environment is make, but other options are possible.

CMake supports \textit{out-of-source build}. It is advised never mix-up source files with generated files. With out-of-source build the whole code structure looks better since all cmake files are hidden in some subdirectory. Another advantage is that it allows to have more build trees for the same source code.
A good practice is to set a cmake file-guard which prevent in-source build. Then we can create build directory and run cmake from it using command \texttt{cmake /path/to/source}.
All cmake files will be located in this build directory.

Compilation can be done in parallel using make: \texttt{make -j [N]}, where \texttt{N} is the number of processes.

In order to specify configuration variables, one can use the command line with the \texttt{-D} option. Very useful to use the \texttt{-i} option for interactive setting of the variables. Another cache editors are cmake-gui and ccmake.

The option \texttt{-L} to cmake allow to see all (non-advanced) variables.

Cmake commands are written in the files CMakeLists.txt.
Cmake allows easy check for the \textbf{existence} of the libraries, functions, header files and compiler flags.

For example, in order to check the existence of the \texttt{-pedantic} flag add to the CMakeLists.txt the following code:

\begin{verbatim}
INCLUDE(CheckCCompilerFlag)
CHECK_C_COMPILER_FLAG(-pedantic C_HAS_PEDANTIC)
IF (C_HAS_PEDANTIC)
  SET(CMAKE_C_FLAGS "${CMAKE_C_FLAGS} -pedantic")
ENDIF ()
\end{verbatim}


CMake provide possibility to specify rules to run at \textbf{install} time. 
To change install directory one should change installation prefix:
\begin{verbatim}
cmake -DCMAKE_INSTALL_PREFIX=/install/path/
\end{verbatim}
Interesting, but CMake does not provide an uninstall option. 



\textbf{CPach} is a cross-platform packaging tool. To setup generator, run for example the following command
\begin{verbatim}
cpack -D CPACK_GENERATOR="ZIP;TGZ" 
\end{verbatim}
Then zip and tgz archive files will be generated in the current build directory.



CMake allow to write \textbf{configuration files}. The function \texttt{configure\_file} reads input file and produce new configure file where all variable definitions from input file replaced by their values specified in CMakeList.txt. If the input file is modified the build system will re-run CMake to re-configure the file and generate the build system again. In our case the input file is cmake\_config.h.in.
If the variable in cmake\_config.h.in. is specified using \texttt{cmakedefine}, then ake will change \texttt{cmakedefine} on \texttt{define} or \texttt{undef} depending on whether the variable defined or not. Another option is to use variables written as \texttt{@var@}. Then CMake will change \texttt{@var@} to the value defined in CMakeFiles.txt.




\section{CTest and CDash}

\subsection{Testing using CTest}

Useful link: \url{http://www.vtk.org/Wiki/CMake/Testing_With_CTest}

The command in the CMakeFiles.txt \texttt{enable\_testing()} adds another build target, which allows to run tests for Makefile generators, or RUN\_TESTS for integrated development environments (like Visual Studio).

% does it just for Makefiles?


\begin{verbatim}
 add_test(NAME <name> [CONFIGURATIONS [Debug|Release|...]]
            [WORKING\_DIRECTORY dir]
            COMMAND <command> [arg1 [arg2 ...]])
\end{verbatim}

It is possible to execute tests in parallel:
\texttt{ctest -j [N]}

\subsection{CDash}

The result of a test run, reformatted for easy review, is called a "dashboard". CDash is an open source, web-based software testing server. One can install own server or use one of public servers.           
The default settings of the module are to submit the dashboard to Kitware's Public Dashboard, where you can register your project for free. Notice that CDash Free Edition doesn't allow more than 10 builds a day.

In order to submit to some other server, put the CTestConfig.cmake in the top level directory of your source code, and set your own dashboard preferences. If you are using a CDash server, you can download a preconfigured file from the respective project page on that server ("Settings" / "Project", tab "Miscellaneous").

There are three types of dashboard submissions:
\begin{description}
 \item[Experimental] means the current state of the project. An experimental submission can be performed at any time, usually interactively from the current working copy of a developer.
  \item[Nightly] is similar to experimental, except that the source tree will be set to the state it was in at a specific nightly time. This ensures that all "nightly" submissions correspond to the state of the project at the same point in time. "Nightly" builds are usually done automatically at a preset time of day.
 \item[Continuous] means that the source tree is updated to the latest revision, and a build / test cycle is performed only if any files were actually updated. Like "Nightly" builds, "Continuous" ones are usually done automatically and repeatedly in intervals.
\end{description}

The standard procedure for runing tests and submutting results to dashboard is
\begin{verbatim}
Start -> Update (only Nightly) -> Configure -> 
Build -> Test -> Coverage -> Submit.
\end{verbatim}


To schedule a nightly build with all options turned on run 
\begin{verbatim}
ctest -VV -S /path/to/my_cdash_script.cmake
\end{verbatim}

\subsection{Code coverage}

Code coverage (\url{http://en.wikipedia.org/wiki/Code_coverage}) it is a measure of how many lines of codes are executed during tests. The information is collected by instrumentation of the code using special programs.

The CTest allow writing scripts which automate testing and submitting dashboards. Cron can run jobs which are running such scripts regularly.

CMake Scripting Of CTest + Cron:
\begin{itemize}
 \item \url{http://www.cmake.org/Wiki/CMake_Scripting_Of_CTest}
\item\url{http://www.gofigure2.org/index.php/developers/2-uncategorised/30-setup-ctest-server.html}
\item\url{https://techbase.kde.org/Development/CMake/DashboardBuilds#How_to_set_up_a_Nightly_build}
\item\url{http://insightsoftwareconsortium.github.io/ITKBarCamp-doc/SoftwareQuality/DashboardSubmission/index.html}
\end{itemize}

Currently coverage is only supported on gcc compiler \texttt{cgov}[\url{http://en.wikipedia.org/wiki/Gcov}]. To perform coverage test, make sure that your code is built with debug symbols, without optimization, and with special flags. These flags are:
\texttt{-fprofile-arcs -ftest-coverage}

CTest maintains the following four metrics:
\begin{itemize}
\item Number of lines covered
\item Number of lines not covered
\item Total number of lines in code
\item Percentage of coverage (number of covered lines/total lines)
\end{itemize}


On the webpage of the project on www.cdash.org one can find all the information about code coverage.


\subsection{Dynamic Analysis}

Dynamic Analysis can be performed using Valgring Memcheck during running tests. Results about memory leaks, uninitialized memory reads, freeing invalid memory, invalid pointer reads and so on can be found on the project dashboard.

\section{CMake for Ergo code}

In order to compile the Ergo code (with Release build type) just run
\begin{verbatim}
cmake /path/to/source
\end{verbatim}


We have bash scripts for testing. Every time CMake copy test directory to the build directory. The reason for that is that test files must know the location of the executable.
%I copy the whole directory (notice that I changed paths in \texttt{test\_hf.sh}) into build. 
After running CMake and compilation, the CTest run tests from the build directory.

CDash for the Ergo code: 

\url{http://my.cdash.org/index.php?project=ergo_cmake}.

The build errors, coverage and dynamic analysis are presented there.  

Notice, that one can make code invisible for public by changing settings of the project on the website.

In order to run coverage test, run cmake with Debug build type: 
\begin{verbatim}
cmake -DCMAKE_BUILD_TYPE=Debug /path/to/source
\end{verbatim}
It will add needed flags to the compiler.
For testing without reporting results one can use just \texttt{make test} or \texttt{ctest} commands.
For testing and reporting results to the dashboard can be used \texttt{cmake -D Experimental}.


\subsection{Cron}

Nigthly tests can be run using cron environment. In order to set up the cron to run nightly test every night at 22:00, one can use the following command
\begin{verbatim}
# m h  dom mon dow   command
00 22 * * *  . ~/.keychain/$HOSTNAME-sh; ctest -VV -S 
    /path/to/DashScript.cmake > /path/to/logs/cdash_nightly.log 2>&1
\end{verbatim}

The DashScript contains commands for CTest to run clone git (or other system) directory, to fetch changes, configure and compile the code, run tests and submit the results to the dashboard.

On order to be able to run such command, it is needed add ssh key to the git account to allow fetching new versions without password check and install \texttt{keychain} to make cron install needed ssh environment variables before running script.

Then after the submission of the job, one can see on dashboard which files were updated.

\section{Using CMake with ninja build system }

CMake by default generates Makefiles on Linux systems, but here is an option to generate files for other build systems. For example, let us consider ninja (\url{https://github.com/martine/ninja}). Ninja is a small build system designed for speed.

\begin{verbatim}
cmake -G Ninja -DCMAKE_MAKE_PROGRAM=/path/to/ninja ..
\end{verbatim}

For compilation of the program just run \texttt{ninja}.
Let us compare results of compilation with make and with ninja with option \texttt{-j4}:

\vskip 10 pt

\begin{tabular}{|l|l|l|l|l|}
	\hline
	& real & user & sys & size\\
	\hline
	cmake + ninja & 5m32.708s & 8m57.294s & 0m7.644s & 3.4MB\\
	cmake + make  & 5m36.039s & 8m48.801s & 0m6.743s & 3.4MB\\
	\hline
\end{tabular}

\vskip 10 pt

Therefore, we did not notice difference in time and size of the executables between ninja and make systems.



\section{Automake vs CMake}

The project Ergo originally use Autotools for generating Makefiles. We generate Makefiles using both CMake and Autotools. Compilation was done with \texttt{-O3} flag and \texttt{-j4} option for make.

The result for Autotools is presented below:

\vskip 10 pt

\begin{tabular}{|l|l|l|l|l|}
	\hline
	& real & user & sys & size\\
	\hline
	autotools + make & 6m1.541s & 6m1.541s & 0m6.501s & 3.2MB\\
	\hline
\end{tabular}

\newpage


Automake vs CMake:

\begin{itemize}
 \item Automake/autoconf are replaced by cmake. Any change in the input config.h.in file or CMakeFiles.txt will result in reconfiguration. Instead if one use autotools, user must call automake or autoconf manually.
 \item CMake provide cache (CMakeCache.txt), where it save all system and user variables. During next configuration, cmake looks at the cache file and add just new variables.  If user modified variables in CMakeFiles.txt, it is needed tp update the cache. It can be done using a CMake GUI (CMakeSetup on Windows or ccmake on UNIX), by wizard mode (cmake -i) or by deleting CMakeCache.txt from the build directory.
  \item  Both generators allow to check existence of functions/include files. CMake allow to specify which information will be written on the configuration file. In contrast, autoconf is adding too much information, which is actually never used in the code.
 \item  In order to compile project with CMake, developer should learn its own non-trivial syntax, different from Autotools.
 \item  Both autotools and CMake is easy to install.
 \item  Autotools are done for Linux systems and work just with Makefiles. CMake provide support for many platforms, compilers and build systems. CMake is perfect for creating portable c++ software.
 \item  Both generators require at least two steps for compiling program. Firstly, you must run cmake/configure and after that use generated makefiles. 
 \item  In addition to a build system, over the years CMake has evolved into a family of development tools: CMake, CTest, CPack, and CDash. Using CTest and CDash it is easy to monitor the status of the development.
\end{itemize}

In conclusion, for any new project it is advisable to use CMake since it is very promising build system which is becoming very popular. In my opinion, CMake allow more control on how to configure/compile the project. 

CTest and CDash tools can be used with different generators of the build system, not just cmake. It provides easy way to configure tests and present the results on the server which are visible to all developers. 

% ctest for other build system generators?
% why ctest
% why cdash




\section{Cmake tutorials}

The documentation can be generated by \texttt{cmake --help-html}.

There are not so many good tutorial about CMake. I am using slides of Eric NOULARD from github (\url{https://github.com/TheErk/CMake-tutorial}) and  slides from the course \url{http://bast.fr/talks/cmake/kung-fu}.

Cmake wiki: \url{http://www.cmake.org/Wiki/CMake}

CMake Useful Variables: \url{http://www.cmake.org/Wiki/CMake_Useful_Variables}



\end{document}







%cmake -DBLAS_LIBRARIES=/opt/OpenBLAS/lib/libopenblas.a -DLAPACK_LIBRARIES=/usr/lib/liblapack.so ..


